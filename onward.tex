%%
%% This is file `sample-sigconf.tex',
%% generated with the docstrip utility.
%%
%% The original source files were:
%%
%% samples.dtx  (with options: `sigconf')
%% 
%% IMPORTANT NOTICE:
%% 
%% For the copyright see the source file.
%% 
%% Any modified versions of this file must be renamed
%% with new filenames distinct from sample-sigconf.tex.
%% 
%% For distribution of the original source see the terms
%% for copying and modification in the file samples.dtx.
%% 
%% This generated file may be distributed as long as the
%% original source files, as listed above, are part of the
%% same distribution. (The sources need not necessarily be
%% in the same archive or directory.)
%%
%% The first command in your LaTeX source must be the \documentclass command.
\documentclass[sigplan,screen]{acmart}

\usepackage{code}
\usepackage{graphicx}
\usepackage{placeins}

%%
%% \BibTeX command to typeset BibTeX logo in the docs
\AtBeginDocument{%
  \providecommand\BibTeX{{%
    \normalfont B\kern-0.5em{\scshape i\kern-0.25em b}\kern-0.8em\TeX}}}

%% Rights management information.  This information is sent to you
%% when you complete the rights form.  These commands have SAMPLE
%% values in them; it is your responsibility as an author to replace
%% the commands and values with those provided to you when you
%% complete the rights form.


%%
%% Submission ID.
%% Use this when submitting an article to a sponsored event. You'll
%% receive a unique submission ID from the organizers
%% of the event, and this ID should be used as the parameter to this command.
%%\acmSubmissionID{123-A56-BU3}

%%
%% The majority of ACM publications use numbered citations and
%% references.  The command \citestyle{authoryear} switches to the
%% "author year" style.
%%
%% If you are preparing content for an event
%% sponsored by ACM SIGGRAPH, you must use the "author year" style of
%% citations and references.
%% Uncommenting
%% the next command will enable that style.
%%\citestyle{acmauthoryear}

%%% The following is specific to Onward! '24-ESSAYS and the paper
%%% '(Programs), Proofs and Refutations (and Tests and Mutants)'
%%% by Alex Groce.
%%%
\setcopyright{acmlicensed}
\acmDOI{10.1145/3689492.3689810}
\acmYear{2024}
\copyrightyear{2024}
\acmISBN{979-8-4007-1215-9/24/10}
\acmConference[Onward! '24]{Proceedings of the 2024 ACM SIGPLAN International Symposium on New Ideas, New Paradigms, and Reflections on Programming and Software}{October 23--25, 2024}{Pasadena, CA, USA}
\acmBooktitle{Proceedings of the 2024 ACM SIGPLAN International Symposium on New Ideas, New Paradigms, and Reflections on Programming and Software (Onward! '24), October 23--25, 2024, Pasadena, CA, USA}
\acmSubmissionID{onward24essays-p7-p}
\received{2024-04-25}
\received[accepted]{2024-08-08}

%%
%% end of the preamble, start of the body of the document source.
\begin{document}

%%
%% The "title" command has an optional parameter,
%% allowing the author to define a "short title" to be used in page headers.
\title{Mutation Driven Development}

\author{Alex Groce}
\orcid{0000-0003-0273-4668}
\affiliation{%
  \institution{Northern Arizona University}
  \city{Flagstaff}
  \country{USA}
}
\email{agroce@gmail.com}

%% By default, the full list of authors will be used in the page
%% headers. Often, this list is too long, and will overlap
%% other information printed in the page headers. This command allows
%% the author to define a more concise list
%% of authors' names for this purpose.
\renewcommand{\shortauthors}{Alex Groce}

%%
%% The abstract is a short summary of the work to be presented in the
%% article.
\begin{abstract}

\end{abstract}

\begin{CCSXML}
<ccs2012>
<concept>
<concept_id>10011007.10010940.10010992.10010998.10011001</concept_id>
<concept_desc>Software and its engineering~Dynamic analysis</concept_desc>
<concept_significance>500</concept_significance>
</concept>
<concept>
<concept_id>10011007.10011074.10011099.10011102.10011103</concept_id>
<concept_desc>Software and its engineering~Software testing and debugging</concept_desc>
<concept_significance>500</concept_significance>
</concept>
</ccs2012>
\end{CCSXML}

\ccsdesc[500]{Software and its engineering~Dynamic analysis}
\ccsdesc[500]{Software and its engineering~Software testing and debugging}

\keywords{software testing software verification, proof, counterexample, tests}



\begin{abstract}
Test driven development (TDD) is a controversial and interesting approach to
software development; while many think of ``better tests'' as a
primary \emph{purpose} of TDD, in practice the goal is as much to use
tests to encourage continued progress in coding.  However, with a
modest change to the process of TDD, the approach can further this
psychological goal, and (it seems likely) end up with a genuniely
better set of tests as well.
  \end{abstract}



\maketitle


\section{Introduction}

The first question I should answer: given the abstract, which sounds
like a fairly standard research proposal, why is this an Onward! Essay
and not a conventional research paper?



\bibliographystyle{ACM-Reference-Format}
\bibliography{bibliography}



\end{document}